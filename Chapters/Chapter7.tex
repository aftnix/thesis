\chapter{Backreaction and Quantum Stability}

\lhead{\emph{Backreaction and Quantum Stability}}
\section{introduction}
Quantized fields, like that we're considering here, in de-Sitter(dS) spacetime lead to particle production. If we consider a thermal spectrum resulting from dS horizon temperature, the energy required to excite these particles reduces the expansion rate, albeit slightly. This in turn, modifies the semi-classical spacetime-metric. This modification of spacetime-metric has the potential to be a cause of instability of the dS manifold as the manifold no longer has constant curvature and loses its time invariance. So the backreaction makes the dS manifold unstable for perturbation. 

\section{Quantum Contribution to dS Manifold}

Classical general relativity dictates, cosmological constant $\Lambda$ has a special property so that the equation of state must satisfy 

\begin{equation}
w = \frac{p}{\rho}=-1
\end{equation}

where $p$ is pressure and $\rho$ is the associated density. This is equivalent to energy-momentum tensor satisfying 

\begin{equation}
T_{\mu\nu}= \Lambda g_{\mu\nu}
\end{equation}

For a positive cosmological constant , negative pressure does negative work as the universe expand. It provides energy to fill the new spacetime volume with cosmological constants. So from this view, the expansion can continue forever. 

Quantum excitations, including gravitons, modify the relation between pressure and energy density, although very slightly. Multi-particle quantum states usually have positive energy density and pressure. That means, the value of $w$ will deviate from $-1$ and the pressure will be insufficient to support $dS$ expansion. 

\section{De-Sitter Thermal spactrum and Backreaction}

In dS spacetime, inertial observer see a thermal distribution of particles and a dS temperature\footnote{See $\S$ 3.5 of Chapter 3 for the priliminary idea.}. Observers who detect thermal particles will not agree with the notion that physical $T_{\mu\nu}$ is proportional to $g_{\mu\nu}$. This deviation violates $dS$ symmetry. \par

It is also argued\autocite{hawking:1977} that the absorption of thermal radiation via back-reaction shrinks the horizon size. The fact that inertial observers in $dS$ spacetime sees a thermal distribution can also be inferred from Unruh effect\footcite{unruh:1976}. One can consider $dS$ spacetime as a time-like hyperboloid embedded in flat spacetime of one higher spatial dimension. From the point of view of embedded spacetime, intertial observers detect a thermal bath. From the perspective of Unruh effect, its evident that the energy of absorbed thermal particles comes from the work done by the accelerating force on the detector. But from $dS$ perspective this energy comes from the work would have been done by negative pressure. So it clearly reduces the amount of expansion. If there were no quantum effect, this particle production and subsequent reduction of expansion would not have been there. \par

The $dS$ temperature is $T=R^-1/2\pi $, where $R$ is the $dS$ radius. The ratio of the thermal energy density to cosmological constant is of the order in Plank units. Therefore it's parametrized by $\epsilon$. The local energy density at late times of expansion, that is after the energy due to particle production appears, is therefore slightly larger than classical case $\rho = \Lambda (1+\epsilon)$. The corresponding pressure $p \approx -\Lambda\eta(1+\xi\epsilon)$ where $\xi=1/3$ and $\xi=0$ correspond to relativistic and non-relativistic thermal particle respectively. Thus if $w\ne -1$, the expansion is no longer exponential. \par

\section{Backreaction}
From Friedman Equation 

\begin{equation}
\frac{\ddot{a}}{a}= -\frac{4\pi G}{3}(\rho + 3p)
\end{equation}

it follows that as long as $\rho>0$  and $w<-1/3$, acceleration is still positive. So we still expect an accelerated expansion of $dS$. The equation of continuity states

\begin{equation}
\dot{\rho} + 3\frac{\dot{a}}{a}(p + \rho) = 0
\end{equation}

From this, using expression of $p$ and $\rho$ stated earlier, it can be shown

\begin{equation}
\frac{\dot{\epsilon}}{\epsilon}+3(1+\xi)\frac{\dot{a}}{a}= 0
\end{equation}

The above equation can readily solved

\begin{equation}
\epsilon \sim a^{-3(1+\xi)}
\end{equation}

\section{Quantum Instability}

As particles produced by earlier expansion are red shifted away, new particles are produced. After many Hubble timescales, the average quantum effect adds up to the thermal bath temperature. Conservation of energy implies that the resulting proper volume of the universe $V$ is slightly smaller than the classical case:

\begin{equation}
V\approx V_{classical} . (1-\epsilon) = exp(3Ht). (1-\epsilon)
\end{equation}
and so
\begin{equation}
\frac{logV}{3t}\approx H - \epsilon/3t
\end{equation}
So the $dS$ will undergo changes after adding these quantum effects and the resulting manifold will not be of constant curvature. At late times, the resulting manifold will differ substantially from classical one. The difference will be of macroscopic order although individual corrections were small. Expansions about the original classical $dS$ spacetime should exhibit instabilities as $dS$ is no longer an exact solution once the back-reaction taken into account.\footcite[It has to be noted stability of $dS$ can be put to question from a point of view different from the one considered Here. For example see][] {ford:1985} \par

The resulting quantum spacetime also cannot be time-reversal invariant. The early and late time geometries can not remain the same if we take the quantum effects into account.\footcite[It has been discussed in detail in   ][] {anderson:2014}