\chapter{The Bogoliubov Transformation and Coefficients}
\lhead{\emph{The Bogoliubov Transformation and Coefficients}}

\section{Introduction}
Bogoliubov transformation\autocite[][(p22)]{carroll:2004} is used to demonstrate the relative nature of vacuum. That means, Whereas in flat space-time, every observer can agree on the vacuum state, its not that obvious in curved space-time. A vacuum is defined as a "no particle" state. That means if the destruction operators of quantum fields acts on vacuum, the particle number remains zero. 

We have to concentrate more on the concept of "particle". It can be shown, that the concept is not at all unambiguous in a presence of gravitation field(Or when an observer resides in a non-inertial frame). To delve into this phenomenon we need to elaborate our concept of vacuum.

In Quantum Theory, there does not exist any "empty space". We can recall from our quantum mechanical study of harmonic oscillator, that the ground state of the system is not zero. A quantum field is conceived as harmonic oscillators at every point of space. So it's evident the ground state of any quantum field, which resides on every point of space, can not be zero. To be more precise, the zero point energies will sum up to be infinite.\autocite[See Chapter 3 of ][]{srednicki:2007}

But we can ignore this infinity. Because only the difference between the vacuum state and any other state is of physical relevance, we don't need to labor over absolute energy content of the vacuum. We can settle by considering vacuum is the lowest  possible energy state. We can elaborate the technical meaning of what we mean by calling something "lowest". One way to deal this is to recall our ladder operators we use to analyze harmonic oscillator systems\footcite{dirac:1958}. In similar fashion we can define creation and annihilation operators. The annihilation operators demotes a state by getting rid of a one particle state from it. Now If some state is thus that we can't get rid of any more particle by applying annihilation operator, we can assume we have reached the lowest energy state, and define that state to be vacuum.

But this definition is observer dependent. As we will find out in the following sections, different observer either engaged in non-inertial motion or residing in a gravitational wave\footnote{Which according to principle of equivalence, can be made indistinguishable given a sufficiently small spacetime region} will disagree on the the definition of vacuum.


\section{Quantum Field in Curved Spacetime}

The Quantum Field we were considering in earlier chapter can be expanded in two different basis 

\begin{equation}
\Phi_{\vec{k}} = \mathlarger{\sum_{\vec{k}}}\bigg[ a_{\vec{k}}\phi_{\vec{k}}+ a_{\vec{k}}^\dagger \phi_{\vec{k}}^*\bigg]= \mathlarger{\sum_{\vec{k}}}\bigg[ b_{\vec{k}}\chi_{\vec{k}}+ b_{\vec{k}}^\dagger \chi_{\vec{k}}^*\bigg]
\end{equation}

where $\phi$ being the De-Sitter scalar field we calculated in the previous chapter and $\chi$ being the orthonormal modes in flat space-time. Also $a^\dagger$ and $b^\dagger$ are creation operators in two basis respectively. Whereas $a$ and $b$ are annihilation operators. These operators obey the usual commutation relations

\begin{gather}
\big[a_{\vec{k}}, a^\dagger_{\vec{k}\textprime}\big]=\big[b_{\vec{k}}, b^\dagger_{\vec{k}\textprime}\big]=\delta ^3(\vec{k}-\vec{k}\textprime) \\
\big[a_{\vec{k}}, a_{\vec{k}\textprime}\big]=\big[a^\dagger_{\vec{k}}, a^\dagger_{\vec{k}\textprime}\big]=\big[b_{\vec{k}}, b_{\vec{k}\textprime}\big]= \big[b^\dagger_{\vec{k}}, b^\dagger_{\vec{k}\textprime}\big] = 0
\end{gather}

Two class of creation and annihilation operators can be written as linear combination of the other.

\begin{equation}
\begin{aligned}
& b_{\vec{k}}=\alpha a_{\vec{k}}+\beta a^\dagger_{\vec{k}\textprime} \\
&b^\dagger_{\vec{k}}=\alpha^* a^\dagger_{\vec{k}}+\beta^* a^\dagger_{\vec{k}\textprime} 
\end{aligned}
\end{equation}

\section{Bogoliubov Transformation}

Now we are ready for the Bogoliubov Transformation. From (3.1), we can infer

\begin{equation*}
\alpha a_{\vec{k}}+\beta a^\dagger_{\vec{k}\textprime} = \alpha^* a^\dagger_{\vec{k}}+\beta^* a^\dagger_{\vec{k}\textprime}  \end{equation*}

Using (3.4) into this equation, we get

\begin{equation*}
a_{\vec{k}}\phi_{\vec{k}}+ a^\dagger_{\vec{}}\phi^*_{\vec{k}}=
\end{equation*}

These transformations from one of basis modes to another is known as the Bogoliubov transformations\footcite[For a review please see ][]{Jacobson:2003} named after Nikolay Bogolyubov, and the coefficients $\alpha$ and $\beta$ implementing the transformation is know as the Bogoliubov coefficients.

\section{Orthonormal Relations}

The invariant inner product between A and B, where Both A and B are functions of x, can defined in curved spacetime as 

\begin{equation}
(A,B) = i \mathlarger{\mathlarger{\int}} \bigg(A*\frac{\partial B}{\partial t} - \frac{\partial A^*}{\partial t}B \bigg) d^3x
\end{equation}

The normalized modes in Minkowski spacetime given by the complete set

\begin{equation}
\chi_{\vec{k}} (x,D) = \frac{1}{(2\pi)^{\frac{D-1}{2}} \sqrt{2|\vec{k}|}}e^{-i(|\vec{k}|t - \vec{k}.\vec{x})}
\end{equation}

Then using the definition of inner product (3.5), we have \footnote{In (3.7) and (3.8) we have used the definition of Dirac delta function
$$\delta^3 (\vec{k} - \vec{k}^\textprime) = \frac{1}{(2\pi)^3}\mathlarger{\int} e^{-i(\vec{k} - \vec{k}^\textprime).\vec{x}}d^3x $$
and the indentity
$$f(y)\delta ^3(y-a) = f(a)\delta ^3(y-a)$$
}

\begin{align}
&(\chi_{\vec{k}}, \chi_{\vec{k}\textprime}) = \delta^3(\vec{k}-\vec{k}^\textprime) \\
&(\chi_{\vec{k}}, \chi_{\vec{k}\textprime})=0
\end{align}

Using (3.5), (3.7) and (3.8) 
\newcommand\numberthis{\addtocounter{equation}{1}\tag{\theequation}}

\begin{align*}
(\chi_{\vec{k}}, \phi_{\vec{k}}) &= (\chi_{\vec{k}}, \alpha\chi_{\vec{k}\textprime}+ \beta ^*\chi^*_{\vec{k}\textprime}\\
&= \alpha (\chi_{\vec{k}}, \chi_{\vec{k}\textprime}) + \beta ^* (\chi_{\vec{k}}, \chi^*_{\vec{k}\textprime}) \\
&= \alpha \delta^3 (\vec{k}-\vec{k}\textprime) \\
&= \alpha_{\vec{k}\vec{k}\textprime} \footnotemark \numberthis
\end{align*}
\footnotetext{For convenience we will be using a shorthand for Dirac delta function. That is for $\delta^3(\vec{k} - \vec{k}\textprime)$, we will be writing $\vec{k}\vec{k}\textprime$ in subscript.}

Similarly,


\begin{equation}
(\chi_{\vec{k}}, \phi^*_{\vec{k}\textprime}) = \beta_{\vec{k}\vec{k}\textprime}
\end{equation}

Also using complex conjugate of (3.7), we can find

\begin{equation}
(\chi^*_{\vec{k}}, \phi_{\vec{k}}) = -\beta*_{\vec{k}\vec{k}\textprime}
\end{equation}

Now, equation (3.9)-(3.11) gives the Bogoliubov Coefficients.

\section{Definition of Vacuum from Different Observer's Point of View}

It has already been pointed out that an empty space from one observer's point of view can be full of particles from another observer's point of view. We now demonstrate this fact using the tools we have just developed.

We're starting with the normalized modes of the quantum filed in  De-Sitter spacetime. Let us denote $\ket{0_\phi}$ is vacuum relative to the annihilation operators defined in De-Sitter spacetime. So by definition, we have

\begin{equation}
a_{\vec{k}}\ket{0_\phi} = 0 \quad \forall \quad \vec{k}
\end{equation}

Also , there will be a vacuum state relative to the Minkowaski operators which we can denote by $\ket{b_\chi}$ and that will obey

\begin{equation}
b_{\vec{k}}\ket{0_\chi}= 0 \quad \forall \quad\vec{k}
\end{equation}

The number operator can be defined for $\ket{\chi}$ vacuum 

\begin{equation}
n_{\chi\vec{k}}= b^\dagger_{\vec{k}}b_{\vec{k}}
\end{equation}

If the system is in $\phi$ vacuum, in which no $\phi$ particle would be observed; we would like to know if any particle can be observed by an observer who is using $\chi$ modes. We therefore calculate the $\chi$ number operator in $\phi$ vacuum

\begin{align*}
\bra{0_\phi}n_{\chi\vec{k}}\ket{0_\phi} &= \bra{0_\phi}b^\dagger_{\vec{k}}b_{\vec{k}}\ket{0_\phi} \\
&=\bra{0_\phi}\bigg( \alpha*a^\dagger_{\vec{k}}+\beta^*a_{\vec{k}}\bigg) \bigg( \alpha a_{\vec{k}}+\beta a^\dagger_{\vec{k}}\bigg ) \ket{0_\phi} \\
&=\beta^* \beta\bra{0_\phi}\ket{0_\phi} \\
&= \beta^*\beta \numberthis
\end{align*}

There is no reason for this number, which is a modules of a complex number, to vanish. So the observer who is using $\chi$ modes, will detect particles whereas the observer using $\phi$ modes will detect none.
