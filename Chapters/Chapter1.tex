\chapter{Introduction}


\lhead{\emph{Introduction}} 

After publishing his famous series summarizing general relativity\footnote{Professor S.N.Bose translated the original GR papers from Einstein and one important paper from Minkowski and published it from Calcutta University. This valuable edition is recently translated in Bengali and Published by Bangla Academy. For original papers, see The Principle of Relativity: Original Papers by A. Einstein and H. Minkowski, University of Calcutta, 1920, pp. 89-163}, Albert Einstein set himself the task to find a solution of his proposed field equation. His aim was to find such solution which will resemble a static, homogeneous and isotropic universe. He did so by considering an universe filled with static, homogeneous pressure-less fluid.\footnote{The smooth pressure-less fluid is usually referred as cosmological dust}. But to sustain $S^3 \times R$ geometry, he had to introduce a strange term, $\Lambda$. This term was introduce to compensate the negative geometrical pressure.Few years later, it was William de Sitter who found an exact solution with constant curvature $R=4\Lambda$.\footcite{de:1917a}\textsuperscript{,}\footcite{de:1917b}

Arthur Eddington later showed\footcite{eddington:1924} by his celebrated coordinate transformation that the de-Sitter solution represent the stationary solution of the whole de-Sitter universe. The de-Sitter Solution exhibited some sort of red-shift and later investigations of Edwin Hubble confirmed that the universe is  indeed  expanding. Friedman's model\footcite{friedman:1999} emphasized the role of the cirtical energy density $3H^2/8\pi G$ with respect to the global evolution of the universe. It stated that the ultimate fate of the universe will depend on the difference between its density to the critical density. In 1930s, with the works of Robertson and Walker, a generalized cosmological model emerged.\footcite[For a review see][]{carroll:2004}

One of the special Robertson-Walker exact solution is the de-Sitter spacetime. The covering manifold can be easily visualized as a 4-dimensional hyperboloid embedded in a flat five-dimensional space. \footcite{hawking:1973}. According to the present knowledge, our is probably approaching a quasi de-Sitter stage in very far future.

In curved spacetime the concept of particle is more subtle than in a flat spacetime. Generally because there is no Lorentz symmetry which can indicate the best vacuum state. A non-static curved spacetime may be responsible for particle creation. Particle production in de-Sitter spacetime was discussed at length in 1960s by Parker.\footcite{parker:1968}\textsuperscript{,}\footcite{parker:1969}\textsuperscript{,}\footcite{parker:1971} He considered a quantum field in a expanding geometry and found that the particle number is not constant. He also found a regularization scheme to account for all the quantum effects generated by his semi-classical approach.\footcite[His personal account of his work on particle creation can consulted for further illumination][]{parker:2012}.

After Guth and Linde developed Inflationary paradigm\footcite{Guth:1980}\textsuperscript{,}\footcite{Linde:1984} to solve long standing cosmological problems like horizon and flatness problems, it also opened a curious area where the phenomenon of particle production due to dynamics of spacetime geometry became relevant. For example, Ford\footnote{ford:1985} argued that the energy comes out of the particle production, when the universe is transitioning from de-Sitter spacetime to matter or radiation dominated universe, is capable of reheating the universe after the inflation.\footnote{Reheating is crucial for the origination of primal matter and radiation of the universe.}

In this study, we have considered a quantum scalar field when the universe was undergoing inflationary stage. We have reviewed literature discussing this phenomenon\footnote{See Bibliography for the complete list} and calculated the energy released because of particle production. We also discussed the phenomenological importance of this energy and reviewed literature to show how this energy can be used for phenomenologically. We discussed the apparent instability quantum fluctuations causes to the classical spacetime. We also discussed some ways this study can broadened in the concluding chapter.