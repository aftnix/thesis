%% ----------------------------------------------------------------
%% Thesis.tex -- MAIN FILE (the one that you compile with LaTeX)
%% ---------------------------------------------------------------- 

% Set up the document
\documentclass[a4paper, 16pt, oneside]{Thesis}  % Use the "Thesis" style, based on the ECS Thesis style by Steve Gunn
\graphicspath{Figures/}  % Location of the graphics files (set up for graphics to be in PDF format)

% Include any extra LaTeX packages required
\usepackage[backend=bibtex,style=verbose]{biblatex}  % Use the "Natbib" style for the references in the Bibliography
\usepackage{verbatim}  % Needed for the "comment" environment to make LaTeX comments
\usepackage{vector}  % Allows "\bvec{}" and "\buvec{}" for "blackboard" style bold vectors in maths
\usepackage{amsmath}
\usepackage[capitalise]{cleveref}
\usepackage{relsize}
\usepackage{flexisym}
\usepackage{chngcntr} 
\usepackage{braket}
\counterwithout{footnote}{chapter}

\hypersetup{urlcolor=blue, colorlinks=true}  % Colours hyperlinks in blue, but this can be distracting if there are many links.
\addbibresource{references.bib}
%% ----------------------------------------------------------------
\begin{document}
\frontmatter      % Begin Roman style (i, ii, iii, iv...) page numbering

% Set up the Title Page
\title  {Thesis Title}
\authors  {\texorpdfstring
            {\href{your web site or email address}{Author Name}}
            {Author Name}
            }
\addresses  {\groupname\\\deptname\\\univname}  % Do not change this here, instead these must be set in the "Thesis.cls" file, please look through it instead
\date       {\today}
\subject    {}
\keywords   {}

\maketitle
%% ----------------------------------------------------------------

\setstretch{1.3}  % It is better to have smaller font and larger line spacing than the other way round

% Define the page headers using the FancyHdr package and set up for one-sided printing
\fancyhead{}  % Clears all page headers and footers
\rhead{\thepage}  % Sets the right side header to show the page number
\lhead{}  % Clears the left side page header

\pagestyle{fancy}  % Finally, use the "fancy" page style to implement the FancyHdr headers

%% ----------------------------------------------------------------
% Declaration Page required for the Thesis, your institution may give you a different text to place here
\clearpage  % Funny Quote page ended, start a new page
%% ----------------------------------------------------------------

% The Abstract Page
\addtotoc{Abstract}  % Add the "Abstract" page entry to the Contents
\thispagestyle{empty} 

\begin{center}
\huge{\emph{Abstract}}
\end{center}
This is a review of the particle production mechanism in early universe when the universe was undergoing exponential expansion. The classical metric during the exponential expansion phase is the de-Sitter metric in its steady state region. A free quantum field is considered in this background and the amount of energy due to the particle production is calculated. Also one of the implications of this energy is reviewed. The implication being the instablity of the classical metric due to quantum fluctuations.




\clearpage  % Abstract ended, start a new page
%% ----------------------------------------------------------------

\setstretch{1.3}  % Reset the line-spacing to 1.3 for body text (if it has changed)

% The Acknowledgements page, for thanking everyone
\acknowledgements{
\addtocontents{toc}{\vspace{1em}}  % Add a gap in the Contents, for aesthetics
I would like to express my gratitude to Dr. Arshad Momen. He not only taught most of what I understand about physics, he indoctrinated me with what can be called "physical way of thinking". He encourage me and others to think more about physical meaning than toiling with mathematics. 

I would also like to acknowlege the course teachers at Theoretical Physics Department, namely, Dr. Golam
Mohammed Bhuiyan, Dr. Md. Tanvir Hanif and Md. Ruhul Amin, for supporting my studies as i struggled to get through because of my engineering background.

Last but not the least i have to acknowledge the assistance of my classmate Tanjib Atique, without whom, i would not have survived the intense coursework

}
\clearpage  % End of the Acknowledgements
%% ----------------------------------------------------------------

\pagestyle{fancy}  %The page style headers have been "empty" all this time, now use the "fancy" headers as defined before to bring them back


%% ----------------------------------------------------------------
\lhead{\emph{Contents}}  % Set the left side page header to "Contents"
\tableofcontents  % Write out the Table of Contents

%% ----------------------------------------------------------------
%% ----------------------------------------------------------------
% End of the pre-able, contents and lists of things
% Begin the Dedication page

% Add a gap in the Contents, for aesthetics


%% ----------------------------------------------------------------
\mainmatter	  % Begin normal, numeric (1,2,3...) page numbering
\pagestyle{fancy}  % Return the page headers back to the "fancy" style

% Include the chapters of the thesis, as separate files
% Just uncomment the lines as you write the chapters

\chapter{Introduction}


\lhead{\emph{Introduction}} 

\section{Historical Background}

After publishing his famous series summarizing general
relativity\footnote{Professor S.N.Bose translated the original GR papers
from Einstein and one important paper from Minkowski and published it from
Calcutta University. This valuable edition is recently translated in
Bengali and Published by Bangla Academy. For original papers, see The
Principle of Relativity: Original Papers by A. Einstein and H. Minkowski,
University of Calcutta, 1920, pp. 89-163}, Albert Einstein set himself the
task to find a solution of his proposed field equation. His aim was to
find such solution which will resemble a static, homogeneous and isotropic
universe. He did so by considering an universe filled with static,
homogeneous pressure-less fluid.\footnote{The smooth pressure-less fluid
is usually referred as cosmological dust}. But to sustain $S^3 \times R$
geometry, he had to introduce a strange term, $\Lambda$. This term was
introduce to compensate the negative geometrical pressure.Few years later,
it was William de Sitter who found an exact solution with constant
curvature
$R=4\Lambda$.\footcite{de:1917a}\textsuperscript{,}\footcite{de:1917b}

Arthur Eddington later showed\footcite{eddington:1924} by his celebrated
coordinate transformation that the de-Sitter solution represent the
stationary solution of the whole de-Sitter universe. The de-Sitter
Solution exhibited some sort of red-shift and later investigations of
Edwin Hubble confirmed that the universe is  indeed  expanding. Friedman's
model\footcite{friedman:1999} emphasized the role of the cirtical energy
density $3H^2/8\pi G$ with respect to the global evolution of the
universe. It stated that the ultimate fate of the universe will depend on
the difference between its density to the critical density. In 1930s, with
the works of Robertson and Walker, a generalized cosmological model
emerged.\footcite[For a review see][]{carroll:2004}

One of the special Robertson-Walker exact solution is the de-Sitter
spacetime. The covering manifold can be easily visualized as
a 4-dimensional hyperboloid embedded in a flat five-dimensional space.
\footcite{hawking:1973}. According to the present knowledge, our is
probably approaching a quasi de-Sitter stage in very far future.

In curved spacetime the concept of particle is more subtle than in a flat
spacetime. Generally because there is no Lorentz symmetry which can
indicate the best vacuum state. A non-static curved spacetime may be
responsible for particle creation. Particle production in de-Sitter
spacetime was discussed at length in 1960s by
Parker.\footcite{parker:1968}\textsuperscript{,}\footcite{parker:1969}\textsuperscript{,}\footcite{parker:1971}
He considered a quantum field in a expanding geometry and found that the
particle number is not constant. He also found a regularization scheme to
account for all the quantum effects generated by his semi-classical
approach.\footcite[His personal account of his work on particle creation
can consulted for further illumination][]{parker:2012}.

After Guth and Linde developed Inflationary
paradigm\footcite{Guth:1980}\textsuperscript{,}\footcite{Linde:1984} to
solve long standing cosmological problems like horizon and flatness
problems, it also opened a curious area where the phenomenon of particle
production due to dynamics of spacetime geometry became relevant. For
example, Ford\footnote{ford:1985} argued that the energy comes out of the
particle production, when the universe is transitioning from de-Sitter
spacetime to matter or radiation dominated universe, is capable of
reheating the universe after the inflation.\footnote{Reheating is crucial
for the origination of primal matter and radiation of the universe.}

\section{Philosophy behind  Quantum Field theory in Curved Spacetime}

After the second world war, a remarkable progress was made in the progress
in the construction of a unified theory of the forces of the nature. The
electromagnetic and weak interactions have received a unified description
with the Wienberg-Salam theory \footnote{Weinberg 1967, Salam 1968 },
while attempts to incorporate the strong interactions as described by
quantum chromodynamics into a wider gauge theory seem to be achieving
success with the so-called grand unified theories\footnote{Georgi&Glashow
1974, for a review see Cline&Mills
        1978}\par

        But from the begining gravity was out of the scheme of things. NOt
        only gravity differs in every other way from other fundamental
        forces of nature, but also it resisted any attempt to fit it
        inside a coherent quantum framework. But that does not mean there
        was any lack of concern. A vigorous attempt to build a quantum
        theory of gravity was always pursuied by some or the
        others\footnote{Isham 1875,1979a,1981}. Even the Scientist like
        Richard Fynman tried to quantize gravity.\footnote{  For a review
        of his attempt see the introduction from some of Fynman's lecture
        of gravitation But a complete quantum theory of gravity remained
        elusive. \par

        In the absense of of a viable theory can one say anything at all
        about the influence of gravitational field on quantum
        phenomenon?In the early days of quantum theory, many calculations
        were undertaken where electromagnetic field was considered as
        a classical background field, interacting with quantized matter.
        Such a semiclassical approximation readily yields some results
        that are in complete accordance with the full theory of quantum
        electrodynamics.\footnote{See for example Shchiff 1949, Chapter
            11} One may therefore hope that a similar regime exists for
            quantum aspects of gravity, in which the gravitational field
            is considered as classical external field, while matter fields
            are fully quantized. If we take Einstein's General Theory of
            relativity as theory of gravity and combine it with the
            principle just stated, we end up with quantum field theory in
            curved spacetime. \par

    
            \section{Scope of a semiclassical theory}
            
        It was orginally pointed out by Plank(1899) that the universal
            constants $G$, $\hbar$ and $c$ could be combined to give new
            fundamental unit of length, the Plank langth
            $(G\hbar/c^3)^{\frac{1}{2}}=1.616\times 10^{-33}cm$, and time,
            the Plank time $(G\hbar/c^5)^{\frac{1}{2}}=5.39\times
                10^{-44}s$. If we treat the gravitational field as a small
                perturbation, and proceed to quantize gravitational field
                in the same manner as we do for QED, the the square of the
                Plank's length appears to play the role of coupling
                constant. But the similarity ends here. Because the
                coupling constant of QED, $e^2/\hbar c$ is dimensionless,
            whereas the Plank length has dimensions. And as the length and
                time scales become comparable with Plank length and time,
            the higher orders becomes as dominant as the lower orders. So
                the concept of small perturbation expansion breaks down.
                \par
                The Plank values therefore mark the frontier at which
                a full theory of Quantum Gravity must be invoked. It
                should be remarked in passing, that prefarably that full
                theory will not be perturbative in nature. As Plank length
                is so small, a semi classical regime has a lot of scope to
                become useful for understanding gravity's effect on
                quantum systems.

                \section{Challenges of Quantum Field theory in Curved
                    Spacetime}


        The reasoning presented in the earlier section, suffers from one
            potential fatal flaw which was ignored in there. Equivalence
            Principle states that all kind of matter and energy couple to
            gravity with equal strength. This includes gravity itself.
            That means gravitational field itself is a source of
            gravitational field. 
The The In this study, we have considered a quantum scalar field when the
universe was undergoing inflationary stage. We have reviewed literature
discussing this phenomenon\footnote{See Bibliography for the complete
list} and calculated the energy released because of particle production.
We also discussed the phenomenological importance of this energy and
reviewed literature to show how this energy can be used for
phenomenologically. We discussed the apparent instability quantum
fluctuations causes to the classical spacetime. We also discussed some
ways this study can broadened in the concluding chapter.
 % Introduction

\input{Chapters/Chapter2} % Background Theory 

\input{Chapters/Chapter3} % Experimental Setup

\chapter{Energy Density Due to Particle Production }

\lhead{\emph{Energy Density Due to Particle Production}}
\section{Introduction}

The energy density due to  scalar field is depends on the number of dimensions we're considering. The D-dimensional solution of Klein-Gordon equation on the steady state region of de-Sitter spacetime can be projected onto the Minkowskian massless modes $\chi$ to compute the corresponding Bogoliubov coefficients $\beta$. The Bogoliubov coefficients are associated with particle production and hence can be used to calculate the energy density due to particle production.


\section{Computation of Bogoliubov coefficients, $\beta$}

The Solution of Klein-Gordon equation in D-dimension as computed in equation (2.24) in $\S$ 2.4

\begin{equation*}
\Phi_{\vec{k}}^{*}(\vec{x},D) = \frac{1}{(2\pi)^{\frac{(D-1}{2}} \sqrt{8H}}e^{-\big(\frac{D-1}{2}\big)Ht}H^{(1,2)}_{\frac{D-1}{2}}\bigg(\frac{|\vec{k}|}{H}e^{-Ht}\bigg)e^{\mp i\vec{k}.\vec{x}}
\end{equation*}

Now the Minkowskian massless modes(see equation (3.6) in  $\S$ 3.4) can be generalized for D dimensions as 

\begin{equation*}
\chi_{\vec{k}} (x,D) = \frac{1}{(2\pi)^{\frac{D-1}{2}} \sqrt{2|\vec{k}|}}e^{-i(|\vec{k}|t - \vec{k}.\vec{x})}
\end{equation*}

And from (3.11) in $\S $3.4, we have 

\begin{equation}
\beta^*_{\vec{k}\vec{k}\textprime} = - (\chi^*_{\vec{k}}, \phi_{\vec{k}\textprime}=-i \mathlarger{\mathlarger{\int}} d^{D-1}x\Bigg[\chi_{\vec{k}}\frac{\partial \Phi_{\vec{k}\textprime}}{\partial t} - \frac{\partial \chi_{\vec{k}}}{\partial}\Phi_{\vec{k}\textprime}\Bigg] 
\end{equation}

Now if substitute (2.24) and (3.6) into (4.1), we have

\begin{align*}
\beta^* = &-i\Bigg( \frac{\pi}{8|\vec{k}|H}\Bigg) \delta^{D-1}(\vec{k}-\vec{k}\textprime)e^{\bigg(\frac{D-1}{2}\bigg)} Ht e^{-i|\vec{k}|t} \\
&\times \Bigg[ \partial_0H^{(1)}_{\frac{D-1}{2}} \Bigg(\frac{|\vec{k}|}{H}e^{-Ht}\Bigg) + \Bigg( i|\vec{k}|-\frac{D-1}{2}H\Bigg) H^{(1)}_{\frac{D-1}{2}}\Bigg(\frac{|\vec{k}|}{H}e^{-Ht}\Bigg)\Bigg] \\
&=-i\Bigg( \frac{\pi}{8|\vec{k}|H}\Bigg) e^{\bigg(\frac{D-1}{2}\bigg)} Ht e^{-i|\vec{k}|t} \\
&\times \Bigg[ \partial_0H^{(1)}_{\frac{D-1}{2}} \Bigg(\frac{|\vec{k}|}{H}e^{-Ht}\Bigg) + \Bigg( i|\vec{k}|-\frac{D-1}{2}H\Bigg) H^{(1)}_{\frac{D-1}{2}}\Bigg(\frac{|\vec{k}|}{H}e^{-Ht}\Bigg)\Bigg] \footnotemark \numberthis
\end{align*}

\footnotetext{The Dirac-delta function enforces the conservation of angular momentum. Therefore it can be easily be suppressed.}

The complex conjugate can readily be found

\begin{align*}
\beta &=+i\Bigg( \frac{\pi}{8|\vec{k}|H}\Bigg) e^{\bigg(\frac{D-1}{2}\bigg)} Ht e^{+i|\vec{k}|t} \\
&\times \Bigg[ \partial_0H^{(1)}_{\frac{D-1}{2}} \Bigg(\frac{|\vec{k}|}{H}e^{-Ht}\Bigg) + \Bigg( -i|\vec{k}|-\frac{D-1}{2}H\Bigg) H^{(2)}_{\frac{D-1}{2}}\Bigg(\frac{|\vec{k}|}{H}e^{-Ht}\Bigg)\Bigg] \footnotemark \numberthis
\end{align*}

\footnotetext{ The following result involving properties of Hankel functions has been used
$$ H^{(1)*}_{\frac{D-1}{2}}\Bigg(\frac{|\vec{k}}{H}e^{-Ht}\Bigg) = H^{(1)*}_{\frac{D-1}{2}}\Bigg(\frac{|\vec{k}}{H}e^{-Ht}\Bigg)$$}

\section{The Number Density of Particles}
The number density of particles with momentum $|\vec{k}|=k$ is defined as

\begin{align}
&number\quad density = \frac{\beta^* \beta}{volume}\\
&n^D_k(t) = \frac{\beta^*(k,D,t)\beta(k,D,t)}{(2\pi)^{D-1}}=\frac{|\beta(k,D,t)|^2}{(2\pi)^{D-1}}
\end{align}
\section{Energy Density Due to Particle Production}

As we are dealing a scalar filed, that means the quantized particles from this field will be spin zero particles, that means they will be Bosons. Bosons obey Bose-Einstein statistics\footcite[][(p221)]{landau:1958}
\begin{equation}
f(k) = \frac{1}{e^{(\frac{k}{k_BT})-1}}
\end{equation}

Where, $k_B$ is the usual Boltzman's constant.

So finally the energy density due to particle production will be

\begin{equation}
\rho(D,t) = \frac{1}{8\pi^2} \mathlarger{\mathlarger{\int}}kn^D_k(t)f(k)d^{D-1}k
\end{equation}

This energy will be released during inflationary period when the universe is expanding exponentially. This phenomenon can be of many use. This energy has the potential to drive crucial transformation of the universe. Example of using this energy for phenomenological purposes will be touched upon in the coming sections. % Experiment 1

\chapter{The Bogoliubov Transformation and Coefficients}
\lhead{\emph{The Bogoliubov Transformation and Coefficients}}

\section{Introduction}
Bogoliubov transformation\autocite[][(p22)]{carroll:2004} is used to demonstrate the relative nature of vacuum. That means, Whereas in flat space-time, every observer can agree on the vacuum state, its not that obvious in curved space-time. A vacuum is defined as a "no particle" state. That means if the destruction operators of quantum fields acts on vacuum, the particle number remains zero. 

We have to concentrate more on the concept of "particle". It can be shown, that the concept is not at all unambiguous in a presence of gravitation field(Or when an observer resides in a non-inertial frame). To delve into this phenomenon we need to elaborate our concept of vacuum.

In Quantum Theory, there does not exist any "empty space". We can recall from our quantum mechanical study of harmonic oscillator, that the ground state of the system is not zero. A quantum field is conceived as harmonic oscillators at every point of space. So it's evident the ground state of any quantum field, which resides on every point of space, can not be zero. To be more precise, the zero point energies will sum up to be infinite.\autocite[See Chapter 3 of ][]{srednicki:2007}

But we can ignore this infinity. Because only the difference between the vacuum state and any other state is of physical relevance, we don't need to labor over absolute energy content of the vacuum. We can settle by considering vacuum is the lowest  possible energy state. We can elaborate the technical meaning of what we mean by calling something "lowest". One way to deal this is to recall our ladder operators we use to analyze harmonic oscillator systems\footcite{dirac:1958}. In similar fashion we can define creation and annihilation operators. The annihilation operators demotes a state by getting rid of a one particle state from it. Now If some state is thus that we can't get rid of any more particle by applying annihilation operator, we can assume we have reached the lowest energy state, and define that state to be vacuum.

But this definition is observer dependent. As we will find out in the following sections, different observer either engaged in non-inertial motion or residing in a gravitational wave\footnote{Which according to principle of equivalence, can be made indistinguishable given a sufficiently small spacetime region} will disagree on the the definition of vacuum.


\section{Quantum Field in Curved Spacetime}

The Quantum Field we were considering in earlier chapter can be expanded in two different basis 

\begin{equation}
\Phi_{\vec{k}} = \mathlarger{\sum_{\vec{k}}}\bigg[ a_{\vec{k}}\phi_{\vec{k}}+ a_{\vec{k}}^\dagger \phi_{\vec{k}}^*\bigg]= \mathlarger{\sum_{\vec{k}}}\bigg[ b_{\vec{k}}\chi_{\vec{k}}+ b_{\vec{k}}^\dagger \chi_{\vec{k}}^*\bigg]
\end{equation}

where $\phi$ being the De-Sitter scalar field we calculated in the previous chapter and $\chi$ being the orthonormal modes in flat space-time. Also $a^\dagger$ and $b^\dagger$ are creation operators in two basis respectively. Whereas $a$ and $b$ are annihilation operators. These operators obey the usual commutation relations

\begin{gather}
\big[a_{\vec{k}}, a^\dagger_{\vec{k}\textprime}\big]=\big[b_{\vec{k}}, b^\dagger_{\vec{k}\textprime}\big]=\delta ^3(\vec{k}-\vec{k}\textprime) \\
\big[a_{\vec{k}}, a_{\vec{k}\textprime}\big]=\big[a^\dagger_{\vec{k}}, a^\dagger_{\vec{k}\textprime}\big]=\big[b_{\vec{k}}, b_{\vec{k}\textprime}\big]= \big[b^\dagger_{\vec{k}}, b^\dagger_{\vec{k}\textprime}\big] = 0
\end{gather}

Two class of creation and annihilation operators can be written as linear combination of the other.

\begin{equation}
\begin{aligned}
& b_{\vec{k}}=\alpha a_{\vec{k}}+\beta a^\dagger_{\vec{k}\textprime} \\
&b^\dagger_{\vec{k}}=\alpha^* a^\dagger_{\vec{k}}+\beta^* a^\dagger_{\vec{k}\textprime} 
\end{aligned}
\end{equation}

\section{Bogoliubov Transformation}

Now we are ready for the Bogoliubov Transformation. From (3.1), we can infer

\begin{equation*}
\alpha a_{\vec{k}}+\beta a^\dagger_{\vec{k}\textprime} = \alpha^* a^\dagger_{\vec{k}}+\beta^* a^\dagger_{\vec{k}\textprime}  \end{equation*}

Using (3.4) into this equation, we get

\begin{equation*}
a_{\vec{k}}\phi_{\vec{k}}+ a^\dagger_{\vec{}}\phi^*_{\vec{k}}=
\end{equation*}

These transformations from one of basis modes to another is known as the Bogoliubov transformations\footcite[For a review please see ][]{Jacobson:2003} named after Nikolay Bogolyubov, and the coefficients $\alpha$ and $\beta$ implementing the transformation is know as the Bogoliubov coefficients.

\section{Orthonormal Relations}

The invariant inner product between A and B, where Both A and B are functions of x, can defined in curved spacetime as 

\begin{equation}
(A,B) = i \mathlarger{\mathlarger{\int}} \bigg(A*\frac{\partial B}{\partial t} - \frac{\partial A^*}{\partial t}B \bigg) d^3x
\end{equation}

The normalized modes in Minkowski spacetime given by the complete set

\begin{equation}
\chi_{\vec{k}} (x,D) = \frac{1}{(2\pi)^{\frac{D-1}{2}} \sqrt{2|\vec{k}|}}e^{-i(|\vec{k}|t - \vec{k}.\vec{x})}
\end{equation}

Then using the definition of inner product (3.5), we have \footnote{In (3.7) and (3.8) we have used the definition of Dirac delta function
$$\delta^3 (\vec{k} - \vec{k}^\textprime) = \frac{1}{(2\pi)^3}\mathlarger{\int} e^{-i(\vec{k} - \vec{k}^\textprime).\vec{x}}d^3x $$
and the indentity
$$f(y)\delta ^3(y-a) = f(a)\delta ^3(y-a)$$
}

\begin{align}
&(\chi_{\vec{k}}, \chi_{\vec{k}\textprime}) = \delta^3(\vec{k}-\vec{k}^\textprime) \\
&(\chi_{\vec{k}}, \chi_{\vec{k}\textprime})=0
\end{align}

Using (3.5), (3.7) and (3.8) 
\newcommand\numberthis{\addtocounter{equation}{1}\tag{\theequation}}

\begin{align*}
(\chi_{\vec{k}}, \phi_{\vec{k}}) &= (\chi_{\vec{k}}, \alpha\chi_{\vec{k}\textprime}+ \beta ^*\chi^*_{\vec{k}\textprime}\\
&= \alpha (\chi_{\vec{k}}, \chi_{\vec{k}\textprime}) + \beta ^* (\chi_{\vec{k}}, \chi^*_{\vec{k}\textprime}) \\
&= \alpha \delta^3 (\vec{k}-\vec{k}\textprime) \\
&= \alpha_{\vec{k}\vec{k}\textprime} \footnotemark \numberthis
\end{align*}
\footnotetext{For convenience we will be using a shorthand for Dirac delta function. That is for $\delta^3(\vec{k} - \vec{k}\textprime)$, we will be writing $\vec{k}\vec{k}\textprime$ in subscript.}

Similarly,


\begin{equation}
(\chi_{\vec{k}}, \phi^*_{\vec{k}\textprime}) = \beta_{\vec{k}\vec{k}\textprime}
\end{equation}

Also using complex conjugate of (3.7), we can find

\begin{equation}
(\chi^*_{\vec{k}}, \phi_{\vec{k}}) = -\beta*_{\vec{k}\vec{k}\textprime}
\end{equation}

Now, equation (3.9)-(3.11) gives the Bogoliubov Coefficients.

\section{Definition of Vacuum from Different Observer's Point of View}

It has already been pointed out that an empty space from one observer's point of view can be full of particles from another observer's point of view. We now demonstrate this fact using the tools we have just developed.

We're starting with the normalized modes of the quantum filed in  De-Sitter spacetime. Let us denote $\ket{0_\phi}$ is vacuum relative to the annihilation operators defined in De-Sitter spacetime. So by definition, we have

\begin{equation}
a_{\vec{k}}\ket{0_\phi} = 0 \quad \forall \quad \vec{k}
\end{equation}

Also , there will be a vacuum state relative to the Minkowaski operators which we can denote by $\ket{b_\chi}$ and that will obey

\begin{equation}
b_{\vec{k}}\ket{0_\chi}= 0 \quad \forall \quad\vec{k}
\end{equation}

The number operator can be defined for $\ket{\chi}$ vacuum 

\begin{equation}
n_{\chi\vec{k}}= b^\dagger_{\vec{k}}b_{\vec{k}}
\end{equation}

If the system is in $\phi$ vacuum, in which no $\phi$ particle would be observed; we would like to know if any particle can be observed by an observer who is using $\chi$ modes. We therefore calculate the $\chi$ number operator in $\phi$ vacuum

\begin{align*}
\bra{0_\phi}n_{\chi\vec{k}}\ket{0_\phi} &= \bra{0_\phi}b^\dagger_{\vec{k}}b_{\vec{k}}\ket{0_\phi} \\
&=\bra{0_\phi}\bigg( \alpha*a^\dagger_{\vec{k}}+\beta^*a_{\vec{k}}\bigg) \bigg( \alpha a_{\vec{k}}+\beta a^\dagger_{\vec{k}}\bigg ) \ket{0_\phi} \\
&=\beta^* \beta\bra{0_\phi}\ket{0_\phi} \\
&= \beta^*\beta \numberthis
\end{align*}

There is no reason for this number, which is a modules of a complex number, to vanish. So the observer who is using $\chi$ modes, will detect particles whereas the observer using $\phi$ modes will detect none.
 % Experiment 2

\chapter{Concluding Remarks}
\lhead{\emph{Concluding Remarks}}

In this study, a free quantum field is considered in a de-Sitter background. The dynamics of the spacetime geometry causes quantum fluctuations which materializes as particles. The accumulated energy in turn influences the geometry itself and which has the potential for causing instability of the de-Sitter spacetime.

The study uses semiclassical approach. That is the gravitational field is not considered quantized but acts as an external classical field. The accumulated energy is added with the classical energy-momentum tensor and with friedman equation, the quantum fluctuations to the spacetime metric is calculated. 


The study finds that the energy which is unleashed by the spacetime metric dynamics, can be accounted for by the Friedman equation. These Quantum fluctuations have the potential to cause macroscopic influence in the evolution of the expanding universe.

As Inflationary scenario is becoming a  more dominant paradigm for thinking about early universe and inflationary scenario involves a exponential expansion phase, the energy produced during this phase by particle production becomes a more and more important accounting tool for different phenomenological studies. Some of the directions at which different proposed research is being done, is also noted and briefly discussed in the study.

The study has two relatively simple constraint. Which can readily be removed. This removal of constraint will broaden the scope of study and also has the potential for unraveling new insights about early universe.

First and most important constraint was the quantum field was considered "free" in the study. That means only way the spacetime metric comes into play is by being and external field. There is no direct coupling of the field with spacetime metric. Some studies\footcite{ford:1987} have proposed an extra term $\epsilon\Phi R$ in the lagrangian, where $R$ is the scalar curvature and $\epsilon$ is a small parameter. That means by including this term, a direct coupling with the spacetime is being conceived. If this is taken into account, then our resulting equation of motion for $\phi$ will no longer be exactly solvable like it has been solved in this study. It has to be solved perturbatively, by usual techniques of quantum field theory. This does not mean we have a quantized theory of gravity, this merely propose a way to couple with the spacetime geometry more directly. Actual quantized theory of gravity will feature creation and annihilation operators corresponding to graviton. If we want to calculate the energy due to particle production when the field has a direct coupling with the gravity, we have to fall-back to perturbation expansion. It is not always obvious if this perturbation will converge. Because constraining $\epsilon$ will be theoretically difficult as we have little empirical clue.

The second constraint is less daunting but it can be suspected that by removing this constraint, nothing very significant will come out. To understand the constraint we have to recall the expression we arrived at the end of Chapter 4 :

\begin{equation*}
\rho(D,t) = \frac{1}{8\pi^2} \mathlarger{\mathlarger{\int}}kn^D_k(t)f(k)d^{D-1}k
\end{equation*}

It is clearly a function of time. So when it will go into the Friedman equation, it should be a function of time. That means, when the Friedman equation will be solved for $H$, the Hubble constant will be a function of time. Now the Hubble constant also appears in the equation of motion of the scalar field $\phi$. But during its solution, $H$ was not considered a function of time. Because of this all time derivative went right through $H$ without changing it. 

To remove this constraint, that means to incorporate the time dependency of Hubble constant into the equation of motion of the quantum field, we have to solve both equation of motion of the scalar field and the Friedman equation simultaneously as coupled differential equation, presumably it has to be done numerically.

It's evident that particle production during exponential expansion phase is of great importance. By broadening the scope, great insights can be achieved of the early phases of the evolution of our universe which is still has not settled into one paradigm. Recent results from precise astronomical observations give us a hint, that we are cancelling out more more models and our description of the early universe is getting more and more accurate. % Results and Discussion

%\input{Chapters/Chapter7} % Conclusion

%% ----------------------------------------------------------------
% Now begin the Appendices, including them as separate files

\addtocontents{toc}{\vspace{2em}} % Add a gap in the Contents, for aesthetics

\appendix % Cue to tell LaTeX that the following 'chapters' are Appendices

\chapter{Derivation of Normalization Constant}
\lhead{\emph{Derivation of Normalization Constant}}
The orthonormal modes in $\S 2.4$ of Chapter 2,

\begin{equation}
\Phi_{\vec{k}} = N_{\vec{k}} e^{\mp \vec{k}.\vec{x}}s ^{\big(\frac{D-1}{2}\big)}H^{(1,2)}_{\frac{D-1}{2}}(s),
\end{equation}

where,

\begin{equation}
s = \frac{|\vec{k}|}{H}e^{-Ht}.
\end{equation}

We need to normalize the modes as 

\begin{equation}
(\Phi_{\vec{k}},\Phi_{\vec{k}}\textprime) \equiv i \mathlarger{\mathlarger{\int}} \sqrt{|g|}d^{D-1}x \Bigg[ \Phi^*_{\vec{k}}\partial_0\Phi_{\vec{k}\textprime} - (\partial_0\Phi^*_{\vec{k}})\Phi_{\vec{k}\textprime}\Bigg] = \delta^{D-1}(\vec{k}-\vec{k}\textprime).
\end{equation}
\clearpage
In our case $\sqrt{g}=e^{(D-1)Ht}$. So we have the inner product

\begin{align*}
(\Phi_{\vec{k}},\Phi_{\vec{k}}\textprime) & = -i|N_{\vec{k}}|^2|\vec{k}|e^{-Ht}e^{(D-1)}s^{(D-1)}\bigg[ H\textprime^{(1)}_{\frac{D-1}{2}}(s) H^{(2)}_{\frac{D-1}{2}}(s) - H^{(1)}_{\frac{D-1}{2}}(s)H\textprime^{(2)}_{\frac{D-1}{2}}(s)\bigg] \\
&\times (2\pi)^{D-1}\delta^{D-1}(\vec{k}-\vec{k}\textprime)\\
&= \delta^{D-1}(\vec{k}-\vec{k}\textprime). \numberthis
\end{align*}\footnote{The $\textprime$ on the Hankel fucntions denotes its derivative with respect to $s$. We have also used the generalized definition of Dirac delta function, namely

$$\delta^{D-1}(\vec{k}-\vec{k}\textprime)= \frac{1}{(2\pi)^{D-1}}\mathlarger{\mathlarger{\int}} e^{-i(\vec{k}-\vec{k}\textprime).\vec{x}}d^{D-1}x.$$

We have also used the following identity
$$f(\omega)\delta^{D-1}(\omega-b)= f(b)\delta^{D-1}(\omega - b)$$
}

We will now use the identity involving Hankel function :

\begin{equation}
H\textprime^{(1)}_{\frac{D-1}{2}}(s) H^{(2)}_{\frac{D-1}{2}}(s) - H^{(1)}_{\frac{D-1}{2}}(s)H\textprime^{(2)}_{\frac{D-1}{2}}(s)
\end{equation}

Therefore using the above result into (A.3) we finally get the normalization constant :

\begin{equation}
|N_{\vec{k}}| = \frac{1}{(2\pi)^{(D-1)/2}\sqrt{8H}}\Bigg( \frac{H}{|\vec{k}|}\Bigg)^{\frac{D-1}{2}}
\end{equation}\label{app:A}	% Appendix Title

%\input{Appendices/AppendixB} % Appendix Title

%\input{Appendices/AppendixC} % Appendix Title

\printbibliography
\lhead{\emph{Bibilography}}

\addtocontents{toc}{\vspace{2em}}  % Add a gap in the Contents, for aesthetics
\backmatter

%% ----------------------------------------------------------------
\end{document}  % The End
%% ----------------------------------------------------------------