\chapter{Massless Scalar Field on de Sitter Background}

\lhead{\emph{Massless Scalar Field on de Sitter Background}} 

\section{Introduction}

As we don't have any quantum theory of gravity, our usual approach to include quantum effects in gravitational systems is semi-classical. That means, we don't quantize the gravitational field itself. Instead we treat the gravitational field as an external classical field and study the dynamics of quantum fields in the presence of the classical field. \par

With this philosophy in mind, we are going to consider a scalar field in a curved spacetime. The usual tools and techniques of quantum field theory in curved spacetime\footcite[For a modern review See][]{holland:2014} are developed with the same "external classical" filed approach. This approach, although bound to be inadequate near Plank limit, nevertheless produces prodigious amount of important results. One of the greatest triumph of this semi-classical approach was when these tools were used to demonstrate that Black-holes do evaporate\footcite[][]{hawking:1974} and kick started one of the most intriguing puzzle of theoretical physics, The Black Hole Information Paradox. \footcite{hawking:1975}

In our case we are considering a quantum field in an expanding universe, presumably in inflationary epoch. The usual De-Sitter solution is a classical solution. We want to see by considering a quantum field in this background, how energy will be produced because of the quantum jitters of the that field and how that accumulated energy will influence the spacetime metric and introduce instabilities.

\section{Conformal Transformation of de Sitter metric}

The Klein Gordon equation is the equation of motion of spin-less particles. We want to solve KG equation in de sitter background. Specifically the steady state region of de sitter space-time. \par

The de-Sitter metric :

\begin{equation}
ds^2 = -dt^2 + e^{Ht}(dr^2 + r^2 d\Omega^2)
\end{equation}


We can map this metric into the conformally flat metric by choosing to do the following 

\begin{equation}\label{eq2}
\begin{split}
ds^2 & = e^{Ht}(-e^{-Ht}dt^2+ dr^2 +r^2 d\Omega^2) \\
&=C^2(\eta)(-d\eta^2 + dr^2 + r^2 d\Omega^2)
\end{split}
\end{equation}

We can see the conformal scale factor $C(\eta)$ is $e^{Ht}$. The relationship between universal time $t$ and conformal time $\eta$ is given by

\begin{equation}
d\eta = e ^{-Ht}dt
\end{equation}

Which leads to 
\begin{equation}
\eta = \frac{1 - e^{-Ht}}{H}
\end{equation}

So this yields 

\begin{equation}
C(\eta) = \frac{1}{1 - \eta H}
\end{equation}

\section{Klein Gordon Equation in de Sitter Metric}

The Klein Gordon Equation in curved Spacetime 

\begin{equation}
\frac{1}{\sqrt{-g}}\partial_\mu(\sqrt{-g}g^{\mu\nu}\partial_\nu\Phi) + m^2\Phi = 0
\end{equation}

The de-Sitter metric in conformal time results in 
\begin{equation}
\sqrt{-g} = C^D(\eta)
\end{equation}

and hence

\begin{equation}\label{eq8}
\begin{split}
& \frac{1}{C^D(\eta)}\bigg[\partial_\mu\bigg(C^D(\eta)\frac{1}{C^2(\eta)}\frac{\partial \Phi}{\partial \eta}\bigg) - \partial_i\bigg(C^D(\eta)\frac{1}{C^2(\eta)}\partial_i \Phi \bigg)\bigg] + m^2\Phi = 0 \\
&\Rightarrow - \frac{1}{C^2(\eta)}\nabla^2\Phi + \frac{1}{C^2(\eta)}\frac{\partial^2 \Phi}{\partial \eta^2} + \frac{D-2}{C(\eta)}\frac{\partial C(\eta)}{\partial \eta}\frac{\partial \Phi}{\partial \eta} + m^2\Phi = 0 \\
&\Rightarrow - \nabla ^2 \Phi + \frac{\partial^2\Phi}{\partial \eta^2}+\frac{D-2}{C(\eta)}\frac{\partial C(\eta)}{\partial \eta}\frac{\partial \Phi}{\partial \eta} + m^2C^2\Phi = 0
\end{split}
\end{equation}

Now, We have

\begin{equation}
\frac{1}{C(\eta)}\frac{\partial C(\eta)}{\partial \eta} = \frac{H}{1-\eta H}
\end{equation}

With this result, The Klein Gordon equation becomes

\begin{equation}
-\nabla^2\Phi + \frac{\partial^2 \Phi}{\partial\eta^2}+\frac{(D-2)H}{(1-\eta H)^2}\Phi=0
\end{equation}
\section{Normalized Modes of the Scalar Field}
The normalized modes of the Scalar Filed can be expressed as

\begin{equation}
\Phi_{\vec{k}}=N_{\vec{k}}e^{-i\vec{k}.\vec{x}}T_{\vec{k}}
\end{equation}

Where $N_{\vec{k}}$ is a normalized constant. With this decomposition, equation (2.11) becomes 

\begin{equation}
\frac{d^2T_{\vec{k}}}{d\eta^2}+\frac{(D-2)H}{1-\eta H} \frac{dT_{\vec{k}}}{d\eta} +\\
\Bigg[|\vec{k}|^2+\frac{m^2}{(1-\eta H)^2)}\Bigg]T_{\vec{k}} = 0
\end{equation}

If we now transform to time-like dimensionless variable $\tau$ such that
\begin{equation}
\tau=1-\eta H=e^{-Ht}\Rightarrow d\tau = -Hd\eta \Rightarrow d\eta = -\frac{1}{H}d\eta
\end{equation}

So our Kleing Gordon equation becomes 

\begin{equation}
\frac{d^2T_{\vec{k}}}{d\tau^2} - \frac{D-2}{\tau}\frac{dT_{\vec{k}}}{d\tau} + \Bigg[\bigg(\frac{|\vec{k}|^2}{H}\bigg)^2 + \bigg(\frac{m}{\tau H}\bigg)^2 \Bigg]T_{\vec{k}}=0
\end{equation}

Defining $s=\frac{|\vec{k}|\tau}{H}$ and $\mu = \frac{m}{H}$ we get
\begin{equation}
\frac{d^2T_{\vec{k}}}{d\tau^2} - \frac{D-2}{\tau}\frac{dT_{\vec{k}}}{ds} + \bigg[1 + \frac{\mu^2}{s^2}\bigg]T_{\vec{k}}=0 
\end{equation}

The equation has the form of a Bessel function. To convert it to a proper Bessel function, we substitute $T_{\vec{k}}= s^\alpha Y_{\vec{k}}$ resulting in
$$
\frac{dT_{\vec{k}}}{ds} = s^\alpha \bigg(\frac{dY_{\vec{k}}}{ds} + \frac{\alpha}{s}Y_{\vec{k}}\bigg)
$$

and
$$
\frac{d^2T_{\vec{k}}}{ds^2} = s^{\alpha} \Bigg[\frac{d^2Y_{\vec{k}}}{ds^2}+\frac{2\alpha}{s}+\frac{\alpha(\alpha - 1)}{s^2}Y_{\vec{k}}\Bigg]
$$

will result in

\begin{equation}
\frac{d^2Y_{\vec{k}}}{ds^2}+\frac{2(\alpha+1)-D}{s}\frac{dY_{\vec{k}}}{ds}+ \bigg[ 1 + \frac{\mu^2-\alpha (D-1-\alpha)}{s^2}\bigg] Y_{\vec{k}}=0
\end{equation}

Now the standard Bessel equation \autocite{arfken2013mathematical} is
\begin{equation}
\frac{d^2Y_{\vec{k}}}{ds^2}+ \frac{1}{s}\frac{dY_{\vec{k}}}{ds}+ \bigg[ 1 - \frac{\nu^2}{s^2}\bigg]=0
\end{equation}

Therefore, to convert equation (2.17) into a Bessel equation, we mast impose



$$2(\alpha+1)-D=1$$
$$\Rightarrow \alpha=\frac{D-1}{2}$$

For the particular case of mass-less field($\mu=0$)

\begin{align}
& \nu^2 = \alpha(D-1-\alpha) \\
\Rightarrow & \nu = \frac{D-1}{2}
\end{align}
This choice of $\alpha$ and non-zero $\mu$, equation (2.16) becomes 

\begin{equation}
\frac{d^2Y_{\vec{k}}}{ds^2}+ \frac{1}{s}\frac{dY_{\vec{k}}}{ds}+ \Bigg[ 1 + \frac{\mu^2 - (\frac{D-1}{2})}{s^2}\Bigg] Y_{\vec{k}}= 0
\end{equation}

Therefore we have our required Bessel equation. Subsequently, in the complex representation, the two linearly independent solutions for the the time dependent amplitude function can be written as combination(for the mass-less case, $\mu=0$)

\begin{equation}
Y_{\vec{k}} = \bigg\{ H^{(1)}_{\frac{D-1}{2}}(s), H^{(2)}_{\frac{D-1}{2}}(s)\bigg\}
\end{equation}

Where $H^{(1,2)}_{\frac{D-1}{2}}(s)$ are the Hankel function.

This gives us

\begin{equation}
T_{\vec{k}}= s^{\big(\frac{D-1}{2}\big)}\bigg\{ H^{(1)}_{\frac{D-1}{2}}(s), H^{(2)}_{\frac{D-1}{2}}(s)\bigg\}
\end{equation}

So we finally have the form of the scalar field
\begin{equation}
\Phi_{\vec{k}} = N_{\vec{k}} e^{\mp \vec{k}.\vec{x}}s ^{\big(\frac{D-1}{2}\big)}H^{(1,2)}_{\frac{D-1}{2}}(s)
\end{equation}
Using the normalization conditions for orthonormal modes, We can calculate the normalization constant. \footnote{See more at $\S$ \cref{app:A}} 

With this value of this constant, the complete solution being
\footnote{
This is a linear combination of two solutions. The combination is expressed with $H^{(1,2)}$. The individual modes are expressed $\phi$ and it's complex conjugate
$$\Phi_{\vec{k}}(\vec{x},D) = \frac{1}{(2\pi)^{\frac{(D-1}{2}} \sqrt{8H}}e^{-\big(\frac{D-1}{2}\big)Ht}H^{(1)}_{\frac{D-1}{2}}\bigg(\frac{|\vec{k}|}{H}e^{-Ht}\bigg)e^{-i\vec{k}.\vec{x}}
$$


and
$$\Phi_{\vec{k}}^{*}(\vec{x},D) = \frac{1}{(2\pi)^{\frac{(D-1}{2}} \sqrt{8H}}e^{-\big(\frac{D-1}{2}\big)Ht}H^{(2)}_{\frac{D-1}{2}}\bigg(\frac{|\vec{k}|}{H}e^{-Ht}\bigg)e^{i\vec{k}.\vec{x}}
$$
}

\begin{equation}
\Phi_{\vec{k}}^{*}(\vec{x},D) = \frac{1}{(2\pi)^{\frac{(D-1}{2}} \sqrt{8H}}e^{-\big(\frac{D-1}{2}\big)Ht}H^{(1,2)}_{\frac{D-1}{2}}\bigg(\frac{|\vec{k}|}{H}e^{-Ht}\bigg)e^{\mp i\vec{k}.\vec{x}}
\end{equation}

Now the modulus of this , that is $\Phi^*\Phi$, denotes the Amplitude $A$.

We are interested in the time variation of this amplitude $A$ in different dimension.These variations are