\chapter{Introduction}


\lhead{\emph{Introduction}} 

\section{Historical Background}

After publishing his famous series summarizing general
relativity\footnote{Professor S.N.Bose translated the original GR papers
from Einstein and one important paper from Minkowski and published it from
Calcutta University. This valuable edition is recently translated in
Bengali and Published by Bangla Academy. For original papers, see The
Principle of Relativity: Original Papers by A. Einstein and H. Minkowski,
University of Calcutta, 1920, pp. 89-163}, Albert Einstein set himself the
task to find a solution of his proposed field equation. His aim was to
find such solution which will resemble a static, homogeneous and isotropic
universe. He did so by considering an universe filled with static,
homogeneous pressure-less fluid.\footnote{The smooth pressure-less fluid
is usually referred as cosmological dust}. But to sustain $S^3 \times R$
geometry, he had to introduce a strange term, $\Lambda$. This term was
introduce to compensate the negative geometrical pressure.Few years later,
it was William de Sitter who found an exact solution with constant
curvature
$R=4\Lambda$.\footcite{de:1917a}\textsuperscript{,}\footcite{de:1917b}

Arthur Eddington later showed\footcite{eddington:1924} by his celebrated
coordinate transformation that the de-Sitter solution represent the
stationary solution of the whole de-Sitter universe. The de-Sitter
Solution exhibited some sort of red-shift and later investigations of
Edwin Hubble confirmed that the universe is  indeed  expanding. Friedman's
model\footcite{friedman:1999} emphasized the role of the cirtical energy
density $3H^2/8\pi G$ with respect to the global evolution of the
universe. It stated that the ultimate fate of the universe will depend on
the difference between its density to the critical density. In 1930s, with
the works of Robertson and Walker, a generalized cosmological model
emerged.\footcite[For a review see][]{carroll:2004}

One of the special Robertson-Walker exact solution is the de-Sitter
spacetime. The covering manifold can be easily visualized as
a 4-dimensional hyperboloid embedded in a flat five-dimensional space.
\footcite{hawking:1973}. According to the present knowledge, our is
probably approaching a quasi de-Sitter stage in very far future.

In curved spacetime the concept of particle is more subtle than in a flat
spacetime. Generally because there is no Lorentz symmetry which can
indicate the best vacuum state. A non-static curved spacetime may be
responsible for particle creation. Particle production in de-Sitter
spacetime was discussed at length in 1960s by
Parker.\footcite{parker:1968}\textsuperscript{,}\footcite{parker:1969}\textsuperscript{,}\footcite{parker:1971}
He considered a quantum field in a expanding geometry and found that the
particle number is not constant. He also found a regularization scheme to
account for all the quantum effects generated by his semi-classical
approach.\footcite[His personal account of his work on particle creation
can consulted for further illumination][]{parker:2012}.

After Guth and Linde developed Inflationary
paradigm\footcite{Guth:1980}\textsuperscript{,}\footcite{Linde:1984} to
solve long standing cosmological problems like horizon and flatness
problems, it also opened a curious area where the phenomenon of particle
production due to dynamics of spacetime geometry became relevant. For
example, Ford\footnote{ford:1985} argued that the energy comes out of the
particle production, when the universe is transitioning from de-Sitter
spacetime to matter or radiation dominated universe, is capable of
reheating the universe after the inflation.\footnote{Reheating is crucial
for the origination of primal matter and radiation of the universe.}

\section{Philosophy behind  Quantum Field theory in Curved Spacetime}

After the second world war, a remarkable progress was made in the progress
in the construction of a unified theory of the forces of the nature. The
electromagnetic and weak interactions have received a unified description
with the Wienberg-Salam theory \footnote{Weinberg 1967, Salam 1968 },
while attempts to incorporate the strong interactions as described by
quantum chromodynamics into a wider gauge theory seem to be achieving
success with the so-called grand unified theories\footnote{Georgi&Glashow
1974, for a review see Cline&Mills
        1978}\par

        But from the begining gravity was out of the scheme of things. NOt
        only gravity differs in every other way from other fundamental
        forces of nature, but also it resisted any attempt to fit it
        inside a coherent quantum framework. But that does not mean there
        was any lack of concern. A vigorous attempt to build a quantum
        theory of gravity was always pursuied by some or the
        others\footnote{Isham 1875,1979a,1981}. Even the Scientist like
        Richard Fynman tried to quantize gravity.\footnote{  For a review
        of his attempt see the introduction from some of Fynman's lecture
        of gravitation But a complete quantum theory of gravity remained
        elusive. \par

        In the absense of of a viable theory can one say anything at all
        about the influence of gravitational field on quantum
        phenomenon?In the early days of quantum theory, many calculations
        were undertaken where electromagnetic field was considered as
        a classical background field, interacting with quantized matter.
        Such a semiclassical approximation readily yields some results
        that are in complete accordance with the full theory of quantum
        electrodynamics.\footnote{See for example Shchiff 1949, Chapter
            11} One may therefore hope that a similar regime exists for
            quantum aspects of gravity, in which the gravitational field
            is considered as classical external field, while matter fields
            are fully quantized. If we take Einstein's General Theory of
            relativity as theory of gravity and combine it with the
            principle just stated, we end up with quantum field theory in
            curved spacetime. \par

    
            \section{Scope of a semiclassical theory}
            
        It was orginally pointed out by Plank(1899) that the universal
            constants $G$, $\hbar$ and $c$ could be combined to give new
            fundamental unit of length, the Plank langth
            $(G\hbar/c^3)^{\frac{1}{2}}=1.616\times 10^{-33}cm$, and time,
            the Plank time $(G\hbar/c^5)^{\frac{1}{2}}=5.39\times
                10^{-44}s$. If we treat the gravitational field as a small
                perturbation, and proceed to quantize gravitational field
                in the same manner as we do for QED, the the square of the
                Plank's length appears to play the role of coupling
                constant. But the similarity ends here. Because the
                coupling constant of QED, $e^2/\hbar c$ is dimensionless,
            whereas the Plank length has dimensions. And as the length and
                time scales become comparable with Plank length and time,
            the higher orders becomes as dominant as the lower orders. So
                the concept of small perturbation expansion breaks down.
                \par
                The Plank values therefore mark the frontier at which
                a full theory of Quantum Gravity must be invoked. It
                should be remarked in passing, that prefarably that full
                theory will not be perturbative in nature. As Plank length
                is so small, a semi classical regime has a lot of scope to
                become useful for understanding gravity's effect on
                quantum systems.

                \section{Challenges of Quantum Field theory in Curved
                    Spacetime}


        The reasoning presented in the earlier section, suffers from one
            potential fatal flaw which was ignored in there. Equivalence
            Principle states that all kind of matter and energy couple to
            gravity with equal strength. This includes gravity itself.
            That means gravitational field itself is a source of
            gravitational field. 
The The In this study, we have considered a quantum scalar field when the
universe was undergoing inflationary stage. We have reviewed literature
discussing this phenomenon\footnote{See Bibliography for the complete
list} and calculated the energy released because of particle production.
We also discussed the phenomenological importance of this energy and
reviewed literature to show how this energy can be used for
phenomenologically. We discussed the apparent instability quantum
fluctuations causes to the classical spacetime. We also discussed some
ways this study can broadened in the concluding chapter.
